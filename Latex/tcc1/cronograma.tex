% CRONOGRAMA-------------------------------------------------------------------

\chapter{CRONOGRAMA}
\label{chap:cronograma}

Este capítulo detalha que etapas serão realizadas ao longo do tempo. O cronograma se refere ao período de um ano: TCC1 e TCC2, onde todas as atividades a serem desenvolvidas devem ser citadas e detalhadas. Segue um exemplo de como o cronograma deve ser estruturado:

% Insere uma linha vertical
\noindent\hrulefill

O desenvolvimento deste trabalho se dará da seguinte forma:

% Atividades
\begin{enumerate}
	\item \label{ela-pro} \textbf{Elaboração da proposta de TCC}: descreva aqui a atividade.
	\item \label{anI} \textbf{Análise dos métodos de localização de regiões candidatas em imagens digitais e
		reconhecimento de caracteres através de redes neurais}: descreva aqui a atividade.
	\item \label{anII} \textbf{Análise do funcionamento da técnica em microcontroladores}: descreva aqui a atividade.
	\item \label{anIII} \textbf{Análise dos modelos implementados em hardware}: descreva aqui a atividade. 
	\item \label{dI} \textbf{Estudo do Linux Embarcado}: descreva aqui a atividade.
	\item \label{dII}  \textbf{Validação dos módulos de localização de regiões candidatas}: descreva aqui a atividade..
	\item \label{dIII} \textbf{Implementação do módulo de captura de imagem}: descreva aqui a atividade. 
	\item \label{esc-tcI}  \textbf{Escrita do TCC1}: descreva aqui a atividade.
	\item \label{imI} \textbf{Implementação da máquina de estados para controle da análise de imagem}: descreva aqui a atividade.
	\item \label{imII} \textbf{Desenvolvimento da camada de integração}: descreva aqui a atividade.
	\item \label{imIII} \textbf{Integração dos módulos que compõe o sistema}: descreva aqui a atividade.
	\item \label{tec} \textbf{Teste e correções}: descreva aqui a atividade.
	\item \label{esc-tcII} \textbf{Escrita do TCC2}: descreva aqui a atividade.
\end{enumerate}

O \autoref{qua:cronograma} mostra o período previsto para as atividades propostas.

% insere o cronograma
% CRONOGRAMA-------------------------------------------------------------------

\chapter{CRONOGRAMA}
\label{chap:cronograma}

Este capítulo detalha que etapas serão realizadas ao longo do tempo. O cronograma se refere ao período de um ano: TCC1 e TCC2, onde todas as atividades a serem desenvolvidas devem ser citadas e detalhadas. Segue um exemplo de como o cronograma deve ser estruturado:

% Insere uma linha vertical
\noindent\hrulefill

O desenvolvimento deste trabalho se dará da seguinte forma:

% Atividades
\begin{enumerate}
	\item \label{ela-pro} \textbf{Elaboração da proposta de TCC}: descreva aqui a atividade.
	\item \label{anI} \textbf{Análise dos métodos de localização de regiões candidatas em imagens digitais e
		reconhecimento de caracteres através de redes neurais}: descreva aqui a atividade.
	\item \label{anII} \textbf{Análise do funcionamento da técnica em microcontroladores}: descreva aqui a atividade.
	\item \label{anIII} \textbf{Análise dos modelos implementados em hardware}: descreva aqui a atividade. 
	\item \label{dI} \textbf{Estudo do Linux Embarcado}: descreva aqui a atividade.
	\item \label{dII}  \textbf{Validação dos módulos de localização de regiões candidatas}: descreva aqui a atividade..
	\item \label{dIII} \textbf{Implementação do módulo de captura de imagem}: descreva aqui a atividade. 
	\item \label{esc-tcI}  \textbf{Escrita do TCC1}: descreva aqui a atividade.
	\item \label{imI} \textbf{Implementação da máquina de estados para controle da análise de imagem}: descreva aqui a atividade.
	\item \label{imII} \textbf{Desenvolvimento da camada de integração}: descreva aqui a atividade.
	\item \label{imIII} \textbf{Integração dos módulos que compõe o sistema}: descreva aqui a atividade.
	\item \label{tec} \textbf{Teste e correções}: descreva aqui a atividade.
	\item \label{esc-tcII} \textbf{Escrita do TCC2}: descreva aqui a atividade.
\end{enumerate}

O \autoref{qua:cronograma} mostra o período previsto para as atividades propostas.

% insere o cronograma
% CRONOGRAMA-------------------------------------------------------------------

\chapter{CRONOGRAMA}
\label{chap:cronograma}

Este capítulo detalha que etapas serão realizadas ao longo do tempo. O cronograma se refere ao período de um ano: TCC1 e TCC2, onde todas as atividades a serem desenvolvidas devem ser citadas e detalhadas. Segue um exemplo de como o cronograma deve ser estruturado:

% Insere uma linha vertical
\noindent\hrulefill

O desenvolvimento deste trabalho se dará da seguinte forma:

% Atividades
\begin{enumerate}
	\item \label{ela-pro} \textbf{Elaboração da proposta de TCC}: descreva aqui a atividade.
	\item \label{anI} \textbf{Análise dos métodos de localização de regiões candidatas em imagens digitais e
		reconhecimento de caracteres através de redes neurais}: descreva aqui a atividade.
	\item \label{anII} \textbf{Análise do funcionamento da técnica em microcontroladores}: descreva aqui a atividade.
	\item \label{anIII} \textbf{Análise dos modelos implementados em hardware}: descreva aqui a atividade. 
	\item \label{dI} \textbf{Estudo do Linux Embarcado}: descreva aqui a atividade.
	\item \label{dII}  \textbf{Validação dos módulos de localização de regiões candidatas}: descreva aqui a atividade..
	\item \label{dIII} \textbf{Implementação do módulo de captura de imagem}: descreva aqui a atividade. 
	\item \label{esc-tcI}  \textbf{Escrita do TCC1}: descreva aqui a atividade.
	\item \label{imI} \textbf{Implementação da máquina de estados para controle da análise de imagem}: descreva aqui a atividade.
	\item \label{imII} \textbf{Desenvolvimento da camada de integração}: descreva aqui a atividade.
	\item \label{imIII} \textbf{Integração dos módulos que compõe o sistema}: descreva aqui a atividade.
	\item \label{tec} \textbf{Teste e correções}: descreva aqui a atividade.
	\item \label{esc-tcII} \textbf{Escrita do TCC2}: descreva aqui a atividade.
\end{enumerate}

O \autoref{qua:cronograma} mostra o período previsto para as atividades propostas.

% insere o cronograma
% CRONOGRAMA-------------------------------------------------------------------

\chapter{CRONOGRAMA}
\label{chap:cronograma}

Este capítulo detalha que etapas serão realizadas ao longo do tempo. O cronograma se refere ao período de um ano: TCC1 e TCC2, onde todas as atividades a serem desenvolvidas devem ser citadas e detalhadas. Segue um exemplo de como o cronograma deve ser estruturado:

% Insere uma linha vertical
\noindent\hrulefill

O desenvolvimento deste trabalho se dará da seguinte forma:

% Atividades
\begin{enumerate}
	\item \label{ela-pro} \textbf{Elaboração da proposta de TCC}: descreva aqui a atividade.
	\item \label{anI} \textbf{Análise dos métodos de localização de regiões candidatas em imagens digitais e
		reconhecimento de caracteres através de redes neurais}: descreva aqui a atividade.
	\item \label{anII} \textbf{Análise do funcionamento da técnica em microcontroladores}: descreva aqui a atividade.
	\item \label{anIII} \textbf{Análise dos modelos implementados em hardware}: descreva aqui a atividade. 
	\item \label{dI} \textbf{Estudo do Linux Embarcado}: descreva aqui a atividade.
	\item \label{dII}  \textbf{Validação dos módulos de localização de regiões candidatas}: descreva aqui a atividade..
	\item \label{dIII} \textbf{Implementação do módulo de captura de imagem}: descreva aqui a atividade. 
	\item \label{esc-tcI}  \textbf{Escrita do TCC1}: descreva aqui a atividade.
	\item \label{imI} \textbf{Implementação da máquina de estados para controle da análise de imagem}: descreva aqui a atividade.
	\item \label{imII} \textbf{Desenvolvimento da camada de integração}: descreva aqui a atividade.
	\item \label{imIII} \textbf{Integração dos módulos que compõe o sistema}: descreva aqui a atividade.
	\item \label{tec} \textbf{Teste e correções}: descreva aqui a atividade.
	\item \label{esc-tcII} \textbf{Escrita do TCC2}: descreva aqui a atividade.
\end{enumerate}

O \autoref{qua:cronograma} mostra o período previsto para as atividades propostas.

% insere o cronograma
\input{dados/quadros/cronograma/cronograma.tex}







