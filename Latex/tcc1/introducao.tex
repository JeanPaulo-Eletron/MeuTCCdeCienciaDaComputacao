% INTRODUÇÃO-------------------------------------------------------------------

\chapter{INTRODUÇÃO}
\label{chap:introducao}

Durante o curso de ciência da computação são transmitidos conceitos interessantes como: “agentes inteligentes”, “algoritmos genéticos”, “inteligência computacional” e “redes neurais”. Também foi possível por meio de uma APS desenvolver um jogo do zero, situações que inspiraram ao tema deste TCC. Com a ajuda de jogos e inteligência artificial é possível aumentar a inclusão dos deficientes visuais aos estudos acadêmicos de todos os tipos e, para o escopo deste trabalho, o tema abordado será o ensino de história através de jogos de texto ditados.

\section{Objetivos}
\label{sec:Objetivos}

\subsection{Gerais}
\label{subsec:Gerais}

\begin{description}
	\setlength\itemindent{15pt}
	\item[•] Realizar levantamento bibliográfico relacionado com a inclusão de deficientes visuais em jogos digitais;

	\item[•] Analisar ferramentas que unidas, possibilitem o desenvolvimento de jogos digitais voltados para o público com deficiências visuais; 

	\item[•] Fortalecer e expandir conceitos de inteligência computacional. 

	\item[•] Contribuir para trabalhos de pesquisa que queiram usar o modelo de conversação automatizada “GPT-3”. 

	\item[•] Atrair índividuos que não são da area da técnologia a conhecer assuntos novos da computação atráves de temas comuns a eles abordado nesse trabalho. 
	
\end{description}  

\subsection{Específicos}
\label{sec:Especificos}
Desenvolver um jogo com o intuito de ajudar deficientes visuais e neurológicos a aprender a diciplina escolar de História, a idéia central é que seja possível ele jogar uma faze de um jogo escrito onde será possível interagir em momentos históricos, para o escopo desse trabalho somente a primeira faze será abordada que será sobre a pré história. 

O principal objetivo é mostrar que é possível criar aplicativos para o ensino de qualquer tipo de matéria visando a inclusão dos deficientes e aumentar.

\section{JUSTIFICATIVA}
\label{sec:Justificativa}

Na atualidade, o mundo está carente de ferramentas gratuitas e pagas que forneçam educação de qualidade para pessoas deficientes ou que possuem défict de atenção, tendo em vista que tais pessoas sentem dificuldade para absorver o conteúdo.1 

Além disso, informações sobre tecnologia da informação no geral, em especial criação e implementação do front end e back end e a inteligência artificial ainda são temas muito carentes de informações ao publico leigo, podendo ser unidas em um projeto ligando as palavras chaves “Jogos” e “Deficientes visuais”, podendo atrair um público até então sem acesso a essa junção. 

Pretende-se também mostrar como foi o caminho de desenvolvimento com ajuda do versionamento de código, utilizando das técnlogias \textbf{GIT} e \textbf{GIT HUB}. 

Segundo (\textbf{citação}), \textbf{GIT} é um sistema de controle de versões
distribuído que pode ser usado para registrar o histórico de edições de qualquer tipo
de arquivo, já \textbf{GIT HUB} é uma plataforma de hospedagem de código-fonte e
arquivos com controle de versão usando o \textbf{GIT}).  

\section{METODOLOGIA}
\label{sec:METODOLOGIA}
Será desenvolvido um sistema web que juntamente com a tecnologia do GPT-3 irá emular um jogo de conversação em linguagem natural, o qual terá como tema um professor de história particular que tenta ensinar seu aluno que possuí défict de atenção por meio do RPG de mesa.   

\section{ORGANIZAÇÃO DO TRABALHO}
\label{sec:organizacaoTrabalho}

Os capítulos desta pesquisa estão organizados da seguinte forma: 

\begin{description}
	\setlength\itemindent{15pt}
	\item[•] Capítulo 1 — Fundamentação teórica. O conceito de ... são descritos no Capítulo 1. 
	
	\item[•] Capítulo 2 — Desenvolvimento. 
	
	\item[•] Capítulo 3 — Testes. 
	
	\item[•] Capítulo 4 — Resultados e conclusão.  
	
\end{description}  

%\simb[texto]{x_i}  

%\sigla[exemplo]{EX}
